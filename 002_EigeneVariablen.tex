% ####################
% # Eigene Variablen #
% ####################

% Der Befehl \newcommand kann auch benutzt werden um Variablen zu definieren:

% Titel der Arbeit:
    \newcommand{\varTitel}{Titel der Arbeit}

% Titel der Arbeit (Hyper-Kompatibel):
    \newcommand{\varTitelHyper}{Hyperref-komp. Name}

% Englischer Titel der Arbeit:
\newcommand{\varTitelEng}{English Title}

% Zur Erlangung eines Akademischen Grades:
    \newcommand{\varAkademGrad}{Zur Erlangung des akademischen Grades\\
    	Master of Science (M.Sc.)\\
    	vorgelegte Thesis}

% Datum:
    \newcommand{\varDate}{2017}

% Abgabetag oder -Datum für die Thesis:
    \newcommand{\varDateII}{XX.YY.2017}
        
% Autoren der Thesis:
    \newcommand{\varAutorEins}{Autor} 

% Ort:
\newcommand{\varOrt}{Ort}

% Erster Betreuer der Thesis:
    \newcommand{\varBetreuer}{Prof. Dr. ??} 
    
% Zweiter Betreuer der Thesis:
    \newcommand{\varZweitgutachter}{Prof. Dr. ??} 
    
% E-Mail-Adressen der Autoren:
    \newcommand{\varEmail}{Mail-Adresse}

% Fachbereich:
    \newcommand{\varFachbereich}{Fachbereich 07: Physik, Mathematik und Informatik, Geographie}

% Institut:
    \newcommand{\varInstitut}{
    I. Physikalisches Institut der Justus-Liebig-Universität Gießen \\
    AG für Atom- und Molekülphysik
    }

% Schlagworte und Keywords:
    \newcommand{\varKeyA}{Master}
    \newcommand{\varKeyB}{Thesis}
    \newcommand{\varKeyC}{Msc}
    \newcommand{\varKeyD}{Messtechnik}
    \newcommand{\varKeyE}{Ionenstoß}
    
% Stil der Einträge im Literaturverzeichnis
    %\newcommand{\varLiteraturLayout}{unsrtdinetal}
     \newcommand{\varLiteraturLayout}{unsrtdinetalDOI}
